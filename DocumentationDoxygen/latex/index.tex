Notre programme permet de jouer aux petits chevaux depuis l\textquotesingle{}interpréteur de commandes à 4 joueurs.~\newline
 Chaque joueur choisit sa couleur et doit sortir ses 4 chevaux en faisant un 6 avec un dé et les faire se déplacer sur le plateau jusqu\textquotesingle{}à la ligne d\textquotesingle{}arrivée pour gagner. La partie se termine quand un joueur positionnne ses 4 chevaux sur la ligne d\textquotesingle{}arrivée.~\newline
~\newline
 Notre programme est loin d\textquotesingle{}être parfait, lorsque les pions arrivent vers leur case finale (avant d\textquotesingle{}atteindre les cases numérotées), il y a quelques problèmes d\textquotesingle{}affichage, malheureusement nous n\textquotesingle{}avons pas trouvé de solutions qui fonctionnent.~\newline
 Lors de la démonstration de notre programme pendant la soutenance, différents scénarios seront montrés.~\newline
~\newline
 D\+E\+R\+O\+U\+L\+E\+M\+E\+NT DU J\+EU \+: Pour sortir un pion de l’écurie et venir le placer sur la première case de son parcours, le joueur doit faire un 6 avec le dé. S’il n’a pas fait 6, il passe son tour. Pendant la partie, il faudra obligatoirement faire un 6 pour mettre un nouveau pion sur la piste. Les règles ci-\/dessous définissent le déplacement des pions sur le plateau.~\newline
 — Une fois un pion sorti, le joueur le fait avancer du nombre de cases égal aux pions du dé. S’il a plusieurs pions sortis, le joueur peut faire avancer, au choix, n’importe lequel de ses pions en piste. ~\newline
— Un joueur qui fait 6 rejoue automatiquement une deuxième fois et peut, au choix, faire sortir un cheval ou avancer de 6 cases. ~\newline
— Un pion dont la route est barrée par un autre pion adverse ne peut se déplacer et passera son tour si les points du dé l’amène au-\/delà du barrage. ~\newline
— En revanche, si le tirage au dé permet à un joueur de placer son pion exactement sur une case déjà occupée par un pion adverse,il renvoie le pion de l’adversaire à l’écurie et prend sa place. Celui qui retourne à l’écurie repart donc de zéro et doit obtenir de nouveau un 6 pour sortir. Toutefois, deux pions de la même couleur peuvent se poser sur la même case.~\newline
Placement des pions dans les cases numérotées\+: A la fin d’un tour complet du jeu, les pions se placent sur les cases numérotées de 1 à 6.\+Chaque pion avance du nombre de cases égal aux points du dé sur la case numérotée correspondante, à condition que celle-\/ci soit libre. Il est également possible de déplacer un pion à l’intérieur des cases numérotées pour libérer de la place et faire entrer un autre pion.\+Si aucun déplacement n’est possible, le pion passe son tour.~\newline
Fin de la partie \+: Le premier joueur qui réussit à ranger ses quatre pions dans les 4 cases numérotées le plus près du centre du plateau de jeu gagne la partie. 